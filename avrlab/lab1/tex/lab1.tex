\documentclass[main.tex]{subfile}

\begin{document}

\section{Materials}
\label{sub:materials}

Below are the materials you'll need for all of the microcontroller labs.

\begin{itemize}
	\item ATMega329p
	\item Pololu USB AVR Programmer
	\item Wiring Kit
	\item Two-Digit 7-Segment Display
	\item 7 pin resistor pack.
	\item 2 Bread boards
	\item $1k \Omega$ and $1.5k \Omega$ resistors
	\item 2 $100 \eta F$ capacitors
	\item $100 \mu H$ inductor
	\item TMP36 tempurature sensor
	\item 6 pin male-female header
\end{itemize}

% section materials (end)

\section{Wiring} 
\label{sec:wiring}

\subsection{Programmer}
\label{sub:programmer}

To wire the ISP programmer use the 6 pin header - see \figref{programmerDia}.

\begin{figure}[H]
	\begin{center}
		\includegraphics[width=\linewidth]{progammerDiagram}
	\end{center}
	\caption{Wiring Diagram for the programmer}
	\label{fig:programmerDia}
\end{figure}

% subsection programmer (end)

\subsection{Wiring the 7 Segment Display}
\label{sub:wiring_the_7_segment_display}

% subsection wiring_the_7_segment_display (end)

Use \figref{wiringDia} to wire your 7-segment display. Note you'll only need to
wire digit 1 of the display for this lab. Next week you'll wire digit 2.

\begin{figure}[h]
	\begin{center}
		\includegraphics[width=\linewidth]{7segWiring}
	\end{center}
	\caption{Wiring the 7-Segment Display: Note use the resistor pack for the resistors.}
	\label{fig:wiringDia}
\end{figure}

% section wiring (end)

\section{Programming} 
\label{sec:programming}

The goal of your program is to get the 7 segment display to count up from $0$ to
$9$ - this way you can confirm that your display works for all digits.

A simple way to do this is to create a function that will take a number and
display it on the 7-segment display by using the GPIO pins. For now just have
this function display the $1$'s place digit on the seven segment display. (Hint: use a
hard coded table to map GPIO outputs to the segments on the display with digits).

Use the documentation for the avrlibc to lookup the variable names of pins and
registers and the datasheet for the ATMega329p to determine the values to use
when setting the registers.

% section programming (end)

\end{document}
